\documentclass[10pt]{article}
\usepackage{graphicx,tabularx,array,geometry,tikz,pgfplots,amsmath,bbding,framed,enumitem,graphicx,wrapfig,multicol,listings,empheq,biblatex}
\usepackage{color} %red, green, blue, yellow, cyan, magenta, black, white
\definecolor{mygreen}{RGB}{28,172,0} % color values Red, Green, Blue
\definecolor{mylilas}{RGB}{170,55,241}
\usepackage[makeroom]{cancel}
\setlength{\parskip}{1ex} %--skip lines between paragraphs
\setlength{\parindent}{0pt} %--don't indent paragraphs
\lstset{language=Matlab,%
  %basicstyle=\color{red},
  breaklines=true,%
  morekeywords={matlab2tikz},
  keywordstyle=\color{blue},%
  morekeywords=[2]{1}, keywordstyle=[2]{\color{black}},
  identifierstyle=\color{black},%
  stringstyle=\color{mylilas},
  commentstyle=\color{mygreen},%
  showstringspaces=false,%without this there will be a symbol in the places where there is a space
  numbers=left,%
  numberstyle={\tiny \color{black}},% size of the numbers
  numbersep=9pt, % this defines how far the numbers are from the text
  emph=[1]{for,end,break},emphstyle=[1]\color{red}, %some words to emphasise
  %emph=[2]{word1,word2}, emphstyle=[2]{style},    
}
\geometry{
 total={210mm,297mm},
 left=15mm,
 right=15mm,
 top=10mm,
 bottom=20mm,
 }
 \pgfplotsset{compat=1.12}
%-- Commands for header
\newcommand*\widefbox[1]{\fbox{\hspace{2em}#1\hspace{2em}}}
\addbibresource{proposal.bib}

\renewcommand{\title}[1]{\textbf{#1}\\}
\renewcommand{\line}{\begin{tabularx}{\textwidth}{X>{\raggedleft}X}\hline\\\end{tabularx}\\[-0.5cm]}
\newcommand{\leftright}[2]{\begin{tabularx}{\textwidth}{X>{\raggedleft}X}#1%
\end{tabularx}\\[-0.5cm]}

%\linespread{2} %-- Uncomment for Double Space
\begin{document}

\title{Notes/Derivations\qquad Amelia Trainer }
\line\\
\leftright{\today}{ }%-- left and right positions in the header
~\par

\subsection*{Introduction}
Due to the overwhelming size and number of necessary nuclear cross section data files needed for a reactor calculation, adopting a multigroup cross section approach is extremely popular for determinisitc neutronics calculations. Cross section resonances greatly impact the flux by creating depressions, since neutrons with energy equal to that of the resonance are highly likely to experience the resonance's corresponding reaction. In most reactors where neutrons are born fast and are slowed down through the resonance range, this effect relating to neutron flux depressions (i.e. self-shielding) is extremely important to characterize.

The generation of accurate multigroup cross sections is central to the success of deterministic reactor physics neutronics calculations. Because of the complexity of nuclear cross section data, which contains large resonances, this process is difficult to perform accurately. The neutron flux exhibits depressions in energy near the resonances, an effect that has both spatial and energy implications known as self-shielding. This self-shielding effect is typically accounted for using an approximate model, such as equivalence in dilution [1, 2] or the subgroup method [3–5]. Recent work has sought to improve the self-shielding approximations and has resulted in new techniques such as the Embedded Self-Shielding Method 



\subsection*{Narrow Resonance Model (homogeneous geometry)}
We start with the Boltzmann Equation.
\begin{equation}\Sigma_{t}(E)\phi(E)=\int_{0}^{\infty}\Sigma_{s}\left(E^{\prime}\rightarrow E\right)\phi\left(E^{\prime}\right)\mathrm{d}E^{\prime}+\frac{\chi(E)}{k_{eff}}\int_{0}^{\infty}v\Sigma_{f}\left(E^{\prime}\right)\phi\left(E^{\prime}\right)\mathrm{d}E^{\prime}\end{equation}

We're working in the resonance region, where scattering is the main form of neutrons slowing down, which allows us to get rid of our fission term, simplifying the Boltzmann Equation to
\begin{equation}\Sigma_{t}(E)\phi(E)=\int_{0}^{\infty}\Sigma_{s}\left(E^{\prime}\rightarrow E\right)\phi\left(E^{\prime}\right)\mathrm{d}E^{\prime}.\end{equation}

We represent the macroscopic cross sections as their constituents: number density $N$, microscopic cross section $\sigma$, and probability $P(E'\rightarrow E)$ of a scattering event resulting in energy transition to $E$ from $E'$.
\begin{equation}\left(\sum\limits_{k}N_{k}\sigma_{t,k}(E)\right)\phi(E)=\sum\limits_{k}\int_{E}^{E/\alpha_{k}}N_{k}\sigma_{s,k}\left(E^{\prime}\right)\phi\left(E^{\prime}\right)P(E'\rightarrow E)\mathrm{d}E^{\prime}\end{equation}
Recalling that 
\begin{equation}P(E'\rightarrow E)dE'=\frac{1}{(1-\alpha_k)E'}dE',\end{equation}
we can further simplify the scattering kernel, bringing the equation to
\begin{equation}\left(\sum\limits_{k}N_{k}\sigma_{t,k}(E)\right)\phi(E)=\sum\limits_{k}\frac{1}{1-\alpha_{k}}\int_{E}^{E/\alpha_{k}}\frac{1}{E'}N_{k}\sigma_{s,k}\left(E^{\prime}\right)\phi\left(E^{\prime}\right)\mathrm{d}E^{\prime}\end{equation}

We now intend to isolate one nuclide as ``resonant'', and letting all others be considered ``non-resonant''\footnote{Note that this does not prevent us from considering multiple nuclides that have resonances. We simply focus on one resonant nuclide at a time.}. The potential scattering cross section is independent of neutron energy. For neutrons with energy in the resonance region that interact with non-resonant nuclides, this is the dominant reaction. Thus, we assume that $\sigma_{t,k\neq r}(E')=\sigma_{s,k\neq r}=\sigma_{pot,k\neq r}$. In other words, we neglect energy dependence and non-scattering reactions, such as absorption, for non-resonant nuclides. 


\begin{align} \left(N_{r}\sigma_{t,r}(E)+\sum\limits_{k\neq r}N_{k}\sigma_{pot,k}\right)\phi(E)=&\frac{1}{1-\alpha_{r}}\int_{E}^{E/\alpha_{r}}\frac{1}{E'}N_{r}\sigma_{s,r}\left(E^{\prime}\right)\phi\left(E^{\prime}\right)dE^{\prime}\\ + \sum\limits_{k\neq r}&\frac{1}{1-\alpha_{k}}\int_{E}^{E/\alpha_{k}}\frac{1}{E'}N_{k}\sigma_{pot,k}\phi\left(E^{\prime}\right)dE^{\prime}\end{align}
  
  
  
We now need to simplify the latter integral, which represents non-resonant scattering. First, we remove all terms without energy dependence out of the integral, yielding
\begin{equation}\sum\limits_{k\neq r}\frac{1}{1-\alpha_{k}}\int_{E}^{E/\alpha_{k}}\frac{1}{E'}N_{k}\sigma_{pot,k}\phi\left(E^{\prime}\right)\mathrm{d}E^{\prime}=\sum\limits_{k\neq r}\frac{N_{k}\sigma_{pot,k}}{1-\alpha_{k}}\int_{E}^{E/\alpha_{k}}\frac{1}{E'}\phi\left(E^{\prime}\right)\mathrm{d}E^{\prime}.\end{equation}
It is clear that without an adequate approximation for $\phi(E')$, we are unable to further simplify this integral. In invoking the Narrow Resonance (NR) approxiamtion, the resonance width of our resonant nuclide $r$is assumed to be narrow compared to the neutron slowing down width. In other words, neutrons are not able to stay in the resonance (the only neutrons that ``see'' the resonance are those that come from higher energies, away from the resonance peak. Additionally, we assume the standard $\phi(E)=1/E$ shape. While it is true that the neutron spectrum surrounding the resonances is not a true $1/E$ shape, the error caused negligible due to the narrow resonance width.


\begin{align}
  \sum\limits_{k\neq r}\frac{1}{1-\alpha_{k}}\int_{E}^{E/\alpha_{k}}\frac{1}{E'}N_{k}\sigma_{pot,k}\phi\left(E^{\prime}\right)\mathrm{d}E^{\prime} = &\sum\limits_{k\neq r}\frac{N_{k}\sigma_{pot,k}}{1-\alpha_{k}}\int_{E}^{E/\alpha_{k}}\frac{1}{(E')^2}\mathrm{d}E^{\prime}\\
  = &\sum\limits_{k\neq r}\frac{N_{k}\sigma_{pot,k}}{1-\alpha_{k}}\left(\frac{1}{E}-\frac{\alpha_k}{E}\right)\\
  = &\sum\limits_{k\neq r}\frac{N_{k}\sigma_{pot,k}}{E}
\end{align}



\begin{equation}\boxed{\sum\limits_{k\neq r}\frac{1}{1-\alpha_{k}}\int_{E}^{E/\alpha_{k}}\frac{1}{E'}N_{k}\sigma_{pot,k}\phi\left(E^{\prime}\right)\mathrm{d}E^{\prime}=\sum\limits_{k\neq r}N_{k}\sigma_{pot,k}\frac{1}{E}}\label{eq:NR-mainConclusion1}\end{equation}

We now insert this approximation of the non-resonant nuclide moderation into our earlier equation, yielding,
\begin{equation}\left(N_{r}\sigma_{t,r}(E)+\sum\limits_{k\neq r}N_{k}\sigma_{pot,k}\right)\phi(E) = \frac{1}{1-\alpha_{r}}\int_{E}^{E/\alpha_{r}}\frac{1}{E'}N_{r}\sigma_{s,r}\left(E^{\prime}\right)\phi\left(E^{\prime}\right)dE^{\prime} + \sum\limits_{k\neq r}\frac{N_{k}\sigma_{pot,k}}{E}\end{equation}
which obviously requires approximations to be made for the remaining integral, which represents neutron moderation via colliding with the resonant nuclide.\par
First, we assume that the resonant nuclide scattering cross section $\sigma_{s,r}(E')$ is adequarely represented using the potential scattering cross section $\sigma_{pot,r}$, which is not energy dependent. Similarly, we assume that the flux is $1/E$, as we assumed for the non-resonant nuclides. This results in

\begin{align}\frac{1}{1-\alpha_{r}}\int_{E}^{E/\alpha_{r}}\frac{1}{E'}N_{r}\sigma_{s,r}\left(E^{\prime}\right)\phi\left(E^{\prime}\right)\mathrm{d}E^{\prime} = &\frac{N_{r}\sigma_{pot,r}}{1-\alpha_{r}}\int_{E}^{E/\alpha_{r}}\frac{1}{E'}\phi\left(E^{\prime}\right)\mathrm{d}E^{\prime} 
  \\
  =&\frac{N_{r}\sigma_{pot,r}}{1-\alpha_{r}}\int_{E}^{E/\alpha_{r}}\frac{1}{E^{\prime}}\frac{\mathrm{d}E^{\prime}}{E^{\prime}}
  \\
  =&N_{r}\sigma_{pot,r}\frac{1}{E}
\end{align}

\begin{equation}\boxed{\frac{1}{1-\alpha_{r}}\int_{E}^{E/\alpha_{r}}\frac{1}{E'}N_{r}\sigma_{s,r}\left(E^{\prime}\right)\phi\left(E^{\prime}\right)\mathrm{d}E^{\prime}=N_{r}\sigma_{pot,r}\frac{1}{E}}\label{eq:NR-mainConclusion2}\end{equation}

These NR approximations to the scattering behavior greatly simplifies the energy dependence of the neutron flux, as shown below.
\begin{equation}\left(N_{r}\sigma_{t,r}(E)+\sum\limits_{k\neq r}N_{k}\sigma_{pot,k}\right)\phi(E) = N_{r}\sigma_{pot,r}\frac{1}{E} + \sum\limits_{k\neq r}\frac{N_{k}\sigma_{pot,k}}{E}\end{equation}

\begin{equation}\phi(E)=\frac{N_{r}\sigma_{pot,r}+\sum\limits_{k\neq r}N_{k}\sigma_{pot,k}}{N_{r}\sigma_{t,r}(E)+\sum\limits_{k\neq r}N_{k}\sigma_{pot,k}}\frac{1}{E}\end{equation}
\begin{equation}\boxed{\phi(E)=\frac{\sigma_{pot,r}+\sigma_{0}}{\sigma_{t,r}(E)+\sigma_{0}}\frac{1}{E}~\mbox{where }\sigma_0=\frac{\sum\limits_{k\neq r}N_k\sigma_{pot,k}}{\Sigma_r}}\end{equation}






\subsection*{Neutron Slowing Down in Heterogeneous Isolated System}
Consider a neutron slowing down in a two-region heterogeneous problem, where $f,m$ represent fuel and moderator, respectively. Note that while the following discussion is based on a two region problem, it can be extended to accommodate more complicated geometries.
\begin{align}\Sigma_{t,f}(E)\phi_{f}(E)V_{f}=P_{f\rightarrow f}(E)V_{f}&\int_{0}^{\infty}\Sigma_{s,f}\left(E^{\prime}\rightarrow E\right)\phi_{f}\left(E^{\prime}\right)dE^{\prime}\\ + P_{m\rightarrow f}(E)V_{m}&\int_{0}^{\infty}\Sigma_{s,m}\left(E^{\prime}\rightarrow E\right)\phi_{m}\left(E^{\prime}\right)dE^{\prime}\end{align}

We separate the macroscopic cross section for scattering to energy $E$ into its number density $N$, microscopic cross section $\sigma_s$, and probability of energy change $P(E'\rightarrow E)$, to rewrite the balance equation as 
\begin{align}\Sigma_{t,f}(E)\phi_{f}(E)V_{f}=P_{f\rightarrow f}(E)V_{f}&\int_{0}^{\infty}\sum\limits_{k\in f}N_k\sigma_{s,k}P(E'\rightarrow E)\phi_{f}\left(E^{\prime}\right)dE^{\prime}\\ + P_{m\rightarrow f}(E)V_{m}&\int_{0}^{\infty}\sum\limits_{k\in m}N_k\sigma_{s,k}P\left(E^{\prime}\rightarrow E\right)\phi_{m}\left(E^{\prime}\right)dE^{\prime}\end{align}

Recalling the energy distribution of a single neutron scattering collision is 
\begin{equation}P(E'\rightarrow E)=\frac{1}{(1-\alpha)E}~\mbox{for }\alpha E\leq E'\leq E,\end{equation}
we can further simplify the balance equation to be

\begin{align}\Sigma_{t,f}(E)\phi_{f}(E)V_{f} = P_{f\rightarrow f}(E)V_{f}\sum\limits_{k\in f}&\int_{E}^{E/\alpha_{k}}\frac{N_{k}\sigma_{s,k}\left(E^{\prime}\right)\phi_{f}\left(E^{\prime}\right)}{\left(1-\alpha_{k}\right)E^{\prime}}dE^{\prime}  \\
+ P_{m\rightarrow f}(E)V_{m}\sum\limits_{k\in m}&\int_{E}^{E/\alpha_{k}}\frac{N_{k}\sigma_{s,k}\left(E^{\prime}\right)\phi_{m}\left(E^{\prime}\right)}{\left(1-\alpha_{k}\right)E^{\prime}}dE^{\prime}.\label{eq:hetero-balance}\end{align}



 Recall Eq.~\ref{eq:NR-mainConclusion1} and Eq.~\ref{eq:NR-mainConclusion2}, where we defined the neutron moderation from non-resonant and resonant nuclides, respectively, as

\begin{equation*}\sum\limits_{k\neq r}\frac{1}{1-\alpha_{k}}\int_{E}^{E/\alpha_{k}}\frac{1}{E'}N_{k}\sigma_{s,k}\phi\left(E^{\prime}\right)\mathrm{d}E^{\prime}=\sum\limits_{k\neq r}N_{k}\sigma_{pot,k}\frac{1}{E}\tag{\ref{eq:NR-mainConclusion1}}\end{equation*}
\begin{equation*}
  \frac{1}{1-\alpha_r}\int_{E}^{E/\alpha_{r}}\frac{1}{E'}N_{r}\sigma_{s,r}(E')\phi\left(E'\right)\mathrm{d}E^{\prime}=N_{r}\sigma_{pot,r}\frac{1}{E}
  \tag{\ref{eq:NR-mainConclusion2}}
\end{equation*}

which can be combined into
\begin{equation}\sum\limits_{k}\frac{1}{1-\alpha_{k}}\int_{E}^{E/\alpha_{k}}\frac{1}{E'}N_{k}\sigma_{pot,k}\phi\left(E^{\prime}\right)\mathrm{d}E^{\prime}=\sum\limits_{k\neq r}\Sigma_{pot,k}\frac{1}{E}\end{equation}
and subsequently plugged into Eq.~\ref{eq:hetero-balance} to yield


\begin{equation}\Sigma_{t,f}(E)\phi_{f}(E)V_{f}=\frac{1}{E}\Big(P_{f\rightarrow f}(E)V_{f}\Sigma_{pot,f}+P_{m\rightarrow f}(E)V_{m}\Sigma_{pot,m}\Big)\end{equation}
\begin{equation}\phi_{f}(E)=\frac{P_{f\rightarrow f}(E)V_f\Sigma_{pot,f}+P_{m\rightarrow f}(E)V_m\Sigma_{pot,m}}{E\Sigma_{t,f}(E)V_f}\end{equation}

Note that while this result is derived using a two-region problem, it can be extended to solve for a flux in region $i\in N$, which is dependent on all $j\in N$ regions:

\begin{equation}\phi_{i}(E)=\frac{1}{E}\Sigma_j\frac{P_{j\rightarrow i}(E)V_{j}\Sigma_{p,j}}{\Sigma_{t,i}(E)V_{i}}\label{eq:resultFromHetero}\end{equation}




\subsection*{Tone's Method}
We start with the result from Eq.~\ref{eq:resultFromHetero}, which represents the energy dependence of the neutron flux in region $i$ as it depends on the collision probabilities between itself and other $j$ regions.
\begin{equation}\phi_{i}(E)=\frac{1}{E}\Sigma_j\frac{P_{j\rightarrow i}(E)V_{j}\Sigma_{p,j}}{\Sigma_{t,i}(E)V_{i}}\end{equation}

The basis of Tone's method lies in the following approximation:
\begin{equation}\frac{P_{j\rightarrow i}(E)}{\Sigma_{t,i}(E)}=\alpha_{i}(E)\frac{P_{j\rightarrow i,g}}{\Sigma_{t,i,g}}\end{equation}
Note what this is actually saying. We're approximating the collision probability and total cross section to be group constants. But to make it slightly better, we're adding in a fine energy term $\alpha_i(E)$, that is \textbf{only dependent on the region we're going into}. This is a major assumption, since in reality the actual collision probability is dependent on other regions (including the source region).

\begin{equation*}\phi_{i}(E)=\frac{1}{E}\alpha_{i}(E)\Sigma_j\frac{P_{j\rightarrow i,g}V_{j}\Sigma_{p,j}}{\Sigma_{t,i,g}V_{i}}\end{equation*}
To achieve a cleaner representation for $\phi_i(E)$, we need a way to better describe $\alpha_i(E)$. To do this, we use two tools:
% We'll use two tools to aid us in the derivation
  \begin{enumerate}
    \item Reciprocity relation
      \begin{equation*}P_{j\rightarrow i}(E)V_{j}\Sigma_{t,j}(E)=P_{i\rightarrow j}(E)V_{i}\Sigma_{t,i}(E)\end{equation*}
        \begin{equation*}P_{i\rightarrow j}(E)=\frac{P_{j\rightarrow i}(E)V_{j}\Sigma_{t,j}(E)}{V_{i}\Sigma_{t,i}(E)}\end{equation*}
    \item Probabilities normalize to 1
      \begin{equation*}\sum\limits_{j}P_{i\rightarrow j}(E)=1\end{equation*}
  \end{enumerate}


Plug reciprocity relation into probabilities requirement
      \begin{equation*}\sum\limits_{j}\left(\frac{P_{j\rightarrow i}(E)V_{j}\Sigma_{t,j}(E)}{V_{i}\Sigma_{t,i}(E)}\right)=1\end{equation*}

      \begin{equation*}\sum\limits_{j}\left(\frac{P_{j\rightarrow i}(E)V_{j}\Sigma_{t,j}(E)}{V_{i}\Sigma_{t,i}(E)}\right)=1\end{equation*}

    Plug in the Tone's approximation
    \begin{equation*}\frac{P_{j\rightarrow i}(E)}{\Sigma_{t,i}(E)}=\alpha_{i}(E)\frac{P_{j\rightarrow i,g}}{\Sigma_{t,i,g}}\end{equation*}
      to yield
      \begin{equation*}\frac{\alpha_i(E)}{V_i\Sigma_{t,i,g}}\sum\limits_{j}\Big(P_{j\rightarrow i,g}V_{j}\Sigma_{t,j}(E)\Big)=1\end{equation*}
        \begin{equation*}\alpha_i(E)=\frac{V_i\Sigma_{t,i,g}}{\sum\limits_{j}\Big(P_{j\rightarrow i,g}V_{j}\Sigma_{t,j}(E)\Big)}\end{equation*}

          We can now plug this definition of $\alpha_i(E)$ into our earlier equation for $\phi_i(E)$.

                \begin{equation*}\phi_{i}(E)=\frac{\alpha_i(E)}{E\Sigma_{t,i,g}V_i}\sum\limits_j\Big(P_{j\rightarrow i,g}V_{j}\Sigma_{p,j}\Big)\end{equation*}

        \begin{equation*}\phi_{i}(E)=\frac{1}{E\Sigma_{t,i,g}V_i}\frac{V_i\Sigma_{t,i,g}}{\sum\limits_{j}\left(P_{j\rightarrow i,g}V_{j}\Sigma_{t,j}(E)\right)}\sum\limits_j\Big(P_{j\rightarrow i,g}V_{j}\Sigma_{p,j}\Big)\end{equation*}
          \begin{equation*}\phi_{i}(E)=\frac{1}{E}\frac{\sum\limits_j\Big(P_{j\rightarrow i,g}V_{j}\Sigma_{p,j}\Big)}{\sum\limits_{j}\left(P_{j\rightarrow i,g}V_{j}\Sigma_{t,j}(E)\right)}\end{equation*}
            \begin{equation*}\phi_{i}(E)\approx\frac{1}{E}\frac{\sum\limits_j\left(P_{j\rightarrow i,g}V_{j}\left(N_{r,j}\sigma_{pot,r}+\sum\limits_{k\neq r}N_{k,j}\sigma_{pot,k}\right)\right)}{\sum\limits_{j}\left(P_{j\rightarrow i,g}V_{j}\left(N_{r,j}\sigma_{r,t}(E)+\sum\limits_{k\neq r}N_{k,j}\sigma_{pot,k}\right)\right)}\end{equation*}
            Note the approximation made above: that the only recognized interaction between neutrons and non-resonant nuclides is potential scattering.
% \begin{equation}\phi_{i}(E)=\frac{1}{E}\alpha_{i}(E)\Sigma_j\frac{P_{j\rightarrow i,g}V_{j}\Sigma_{p,j}}{\Sigma_{t,i,g}V_{i}}\end{equation}

% \begin{equation}~\end{equation}
% Another tool that we're going to use is the reciprocity theorem
% \begin{equation}P_{j\rightarrow i}(E)V_{j}\Sigma_{t,j}(E)=P_{i\rightarrow j}(E)V_{i}\Sigma_{t,i}(E)\end{equation}

% Finally, normalization requirement for collision probabilities is also considered.
% \begin{equation}\Sigma_{j}P_{i\rightarrow j}(E)=1\end{equation}


% \begin{equation}1=\Sigma_{j}P_{i\rightarrow j}(E)\end{equation}

% \begin{equation}1=\Sigma_j\frac{P_{j\rightarrow i}(E)V_{j}\Sigma_{t,j}(E)}{V_{i}\Sigma_{t,i}(E)}\end{equation}

% \begin{equation}1=\Sigma_j\alpha_{i}(E)\frac{P_{j\rightarrow i,g}}{\Sigma_{t,i,g}}\frac{V_{j}\Sigma_{t,j}(E)}{V_{i}}\end{equation}



% \begin{equation}1=\alpha_{i}(E)\frac{1}{\Sigma_{t,i,g}V_{i}}\Sigma_jP_{j\rightarrow i,g}V_{j}\Sigma_{t,j}(E)\end{equation}

% \begin{equation}\alpha_{i}(E)=\frac{\Sigma_{t,i,g}V_{i}}{\Sigma_jP_{j\rightarrow i,g}V_{j}\Sigma_{t,j}(E)}\end{equation}

\begin{equation}\phi_{i}(E)=\frac{1}{E}\alpha_{i}(E)\Sigma_j\frac{P_{j\rightarrow i,g}V_{j}\sum\limits_{p,j}}{\Sigma_{t,i,g}V_{i}}\end{equation}

\begin{equation}=\frac{1}{E}\frac{\Sigma_{t,i,g}V_{i}}{\Sigma_jP_{j-i,g}V_{t,j}(E)}\Sigma_j\frac{P_{j-i,g}V_{j}\Sigma_{p,j}}{\Sigma_{t,i,g}V_{i}}\end{equation}

\begin{equation}=\frac{1}{E}\frac{\Sigma_jP_{j\rightarrow i,g}V_{j}\Sigma_{p,j}}{\Sigma_jP_{j\rightarrow i,g}V_{j}\Sigma_{t,j}(E)}\end{equation}

\begin{equation}\cong\frac{\Sigma_jP_{j\rightarrow i,g}V_{j}\cdot\left(N_{r,j}\sigma_{pot,r}+\sum\limits_{k\neq r}N_{k,j}\sigma_{pot,k}\right)}{\Sigma_jP_{j\rightarrow i,g}V_{j}\cdot\left(N_{r,j}\sigma_{t,r}(E)+\sum\limits_{k\neq r}N_{k,j}\sigma_{pot,k}\right)}\end{equation}

\begin{equation}=\frac{1}{E}\frac{\sigma_{pot,r}\Sigma_jP_{j\rightarrow i,g}V_{j}N_{r,j}+\Sigma_jP_{j\rightarrow i,g}V_{j}\sum\limits_{k\neq r}N_{k,j}\sigma_{pot,k}}{\sigma_{t,r}(E)\Sigma_jP_{j\rightarrow i,g}V_{j}N_{r,j}+\Sigma_jP_{j\rightarrow i,g}V_{j}\sum\limits_{k\neq r}N_{k,j}\sigma_{pot,k}}\end{equation}

\begin{equation}=\frac{1}{E}\frac{\sigma_{pot,r}+\left(\sum\limits_j P_{j\rightarrow i,g}V_{j}\sum\limits_{k\neq r}N_{k,j}\sigma_{pot,k}\right)/\left(\sum\limits_jP_{j\rightarrow i,g}V_{j}N_{r,j}\right)}{\sigma_{t,r}(E)+\left(\sum\limits_j P_{j\rightarrow i,g}V_{j}\sum\limits_{k\neq r}N_{k,j}\sigma_{pot,k}\right)/\left(\Sigma_jP_{j\rightarrow i,g}V_{j}N_{r,j}\right)}\end{equation}

\begin{equation}=\frac{1}{E}\frac{\sigma_{pot,r}+\sigma_{0}}{\sigma_{t,r}(E)+\sigma_{0}}\end{equation}


\begin{equation}\sigma_{0}=\frac{\Sigma_j\sum\limits_{k\neq r}P_{j\rightarrow i,g}V_{j}N_{k,j}\sigma_{pot,k}}{\Sigma_jP_{j\rightarrow i,g}V_{j}N_{r,j}}\end{equation}




%\printbibliography

\end{document}















