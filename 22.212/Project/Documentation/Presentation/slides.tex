\documentclass{beamer}
%
% Choose how your presentation looks.
%
% For more themes, color themes and font themes, see:
% http://deic.uab.es/~iblanes/beamer_gallery/index_by_theme.html
%
\mode<presentation>
{
  %\usetheme{CambridgeUS}      % or try Darmstadt, Madrid, Warsaw, ...
  \usetheme{Dresden}      % or try Darmstadt, Madrid, Warsaw, ...
  \usecolortheme{beaver} % or try albatross, beaver, crane, ...
  \usefonttheme{default}  % or try serif, structurebold, ...
  \setbeamertemplate{navigation symbols}{}
  \setbeamertemplate{caption}[numbered]
  \setbeamertemplate{footline}[frame number]{}
  %\setbeamertemplate{navigation symbols}{}
  \setbeamersize{text margin left=5mm,text margin right=5mm} 



} 


\usepackage[english]{babel}
\usepackage[utf8x]{inputenc}

\title[Tone’s method]{Tone’s method for resonance self-shielding}
\author{Amelia Trainer}
%\institute{Where You're From}
\date{Fall 2018 22.212}

\begin{document}

\begin{frame}
  \titlepage
\end{frame}

% Uncomment these lines for an automatically generated outline.
\begin{frame}{Outline}
  \tableofcontents
\end{frame}

\section{Introduction}

\begin{frame}{Introduction}
\begin{figure}
\includegraphics[width=0.8\textwidth]{f1}
\end{figure}
\end{frame}




\begin{frame}
\begin{figure}
\includegraphics[width=0.75\textwidth]{self-shielding}
  \caption{Energy dependent neutron flux versus fuel temperature at 6.67eV resonance of 238U [nuclear-power.net].}
\end{figure}
\end{frame}




\begin{frame}

\section{Background}
\subsection{Homogeneous Slowing Down}
  \textbf{Homogeneous Slowing Down}

We start with the Boltzmann Equation.
  \begin{align*}\Sigma_{t}(E)\phi(E)=&\int_{0}^{\infty}\Sigma_{s}\left(E^{\prime}\rightarrow E\right)\phi\left(E^{\prime}\right)\mathrm{d}E^{\prime}\\+\frac{\chi(E)}{k_{eff}}&\int_{0}^{\infty}v\Sigma_{f}\left(E^{\prime}\right)\phi\left(E^{\prime}\right)\mathrm{d}E^{\prime}\end{align*}

Elastic down-scattering is the dominant interaction here, allowing us to eliminate fission term
\begin{equation*}\Sigma_{t}(E)\phi(E)=\int_{0}^{\infty}\Sigma_{s}\left(E^{\prime}\rightarrow E\right)\phi\left(E^{\prime}\right)\mathrm{d}E^{\prime}.\end{equation*}

  Split the macroscopic cross section into its components

  \begin{equation*}\left(\sum\limits_{k}N_{k}\sigma_{t,k}(E)\right)\phi(E)=\sum\limits_{k}\int_{E}^{E/\alpha_{k}}N_{k}\sigma_{s,k}\left(E^{\prime}\right)\phi\left(E^{\prime}\right)P(E'\rightarrow E)\mathrm{d}E^{\prime}\end{equation*}
\end{frame}
\begin{frame}
Recall that 
\begin{equation*}P(E'\rightarrow E)dE'=\frac{1}{(1-\alpha_k)E'}dE',\end{equation*}
  we simplify the scattering term
\begin{equation*}\left(\sum\limits_{k}N_{k}\sigma_{t,k}(E)\right)\phi(E)=\sum\limits_{k}\frac{1}{1-\alpha_{k}}\int_{E}^{E/\alpha_{k}}\frac{1}{E'}N_{k}\sigma_{s,k}\left(E^{\prime}\right)\phi\left(E^{\prime}\right)\mathrm{d}E^{\prime}\end{equation*}

Separate resonant nuclide from non-resonant nuclides, and represent non-resonant nuclides using only the potential scattering cross section.

  \begin{align*} \left(N_{r}\sigma_{t,r}(E)+\sum\limits_{k\neq r}N_{k}\sigma_{pot,k}\right)\phi(E)=&\frac{1}{1-\alpha_{r}}\int_{E}^{E/\alpha_{r}}\frac{N_{r}\sigma_{s,r}\left(E^{\prime}\right)\phi\left(E^{\prime}\right)}{E'}dE^{\prime}\\ + \sum\limits_{k\neq r}&\frac{1}{1-\alpha_{k}}\int_{E}^{E/\alpha_{k}}\frac{N_{k}\sigma_{pot,k}\phi\left(E^{\prime}\right)}{E'}dE^{\prime}\end{align*}
  
\end{frame}



\begin{frame}
  \textbf{Narrow Resonance Approxiamtion}
  A sufficiently thin resonance allows us to approximate that every scattering event will miss the resonance. We thus assume that the scattering kernel is simply equal to the potential scattering cross section $\sigma_{pot}$, which is constant in energy.
\end{frame}




\begin{frame}
  
We now need to simplify the latter integral, which represents \textbf{scattering contributions of the non-resonant nuclides}. First, we remove all terms without energy dependence out of the integral, yielding
  \begin{align*}\mbox{Non-res scattering}=&\sum\limits_{k\neq r}\frac{1}{1-\alpha_{k}}\int_{E}^{E/\alpha_{k}}\frac{1}{E'}N_{k}\sigma_{pot,k}\phi\left(E^{\prime}\right)\mathrm{d}E^{\prime} \\=&\sum\limits_{k\neq r}\frac{N_{k}\sigma_{pot,k}}{1-\alpha_{k}}\int_{E}^{E/\alpha_{k}}\frac{1}{E'}\phi\left(E^{\prime}\right)\mathrm{d}E^{\prime}
 %   \\=&\sum\limits_{k\neq r}\frac{N_{k}\sigma_{pot,k}}{1-\alpha_{k}}\int_{E}^{E/\alpha_{k}}\frac{1}{(E')^2}\mathrm{d}E^{\prime}
  \\= &\sum\limits_{k\neq r}\frac{N_{k}\sigma_{pot,k}}{1-\alpha_{k}}\left(\frac{1}{E}-\frac{\alpha_k}{E}\right)
  \\= &\sum\limits_{k\neq r}\frac{N_{k}\sigma_{pot,k}}{E}
\end{align*}

  \begin{equation*}\boxed{\sum\limits_{k\neq r}\frac{1}{1-\alpha_{k}}\int_{E}^{E/\alpha_{k}}\frac{1}{E'}N_{k}\sigma_{pot,k}\phi\left(E^{\prime}\right)\mathrm{d}E^{\prime}=\sum\limits_{k\neq r}N_{k}\sigma_{pot,k}\frac{1}{E}}\label{eq:NR-mainConclusion1}\end{equation*}
\end{frame}

\begin{frame}
  
We now follow similar steps to simplify the \textbf{scattering contributions of the resonant nuclide}. First, we remove all terms without energy dependence out of the integral, yielding
  \begin{align*}\mbox{Res scattering}=&\frac{1}{1-\alpha_{r}}\int_{E}^{E/\alpha_{r}}\frac{1}{E'}N_{r}\sigma_{s,r}\left(E^{\prime}\right)\phi\left(E^{\prime}\right)\mathrm{d}E^{\prime} \\
    =& \frac{N_{r}\sigma_{pot,r}}{1-\alpha_{r}}\int_{E}^{E/\alpha_{r}}\frac{1}{E'}\phi\left(E^{\prime}\right)\mathrm{d}E^{\prime}
  \\= &\frac{N_{r}\sigma_{pot,r}}{1-\alpha_{r}}\left(\frac{1}{E}-\frac{\alpha_r}{E}\right)
  \\
    =&\frac{N_{r}\sigma_{pot,r}}{E}
\end{align*}


  \begin{equation*}\boxed{\frac{1}{1-\alpha_{r}}\int_{E}^{E/\alpha_{r}}\frac{1}{E'}N_{r}\sigma_{pot,r}\phi\left(E^{\prime}\right)\mathrm{d}E^{\prime}=N_{r}\sigma_{pot,r}\frac{1}{E}}\end{equation*}
\end{frame}

\begin{frame}
  \begin{equation*}{\sum\limits_{k\neq r}\frac{1}{1-\alpha_{k}}\int_{E}^{E/\alpha_{k}}\frac{1}{E'}N_{k}\sigma_{pot,k}\phi\left(E^{\prime}\right)\mathrm{d}E^{\prime}=\sum\limits_{k\neq r}N_{k}\sigma_{pot,k}\frac{1}{E}}\label{eq:NR-mainConclusion1}\end{equation*}
  \begin{equation*}{\frac{1}{1-\alpha_{r}}\int_{E}^{E/\alpha_{r}}\frac{1}{E'}N_{r}\sigma_{pot,r}\phi\left(E^{\prime}\right)\mathrm{d}E^{\prime}=N_{r}\sigma_{pot,r}\frac{1}{E}}\end{equation*}
    \[~\]
    Putting these together
  \begin{equation*}{\sum\limits_{k}\frac{1}{1-\alpha_{k}}\int_{E}^{E/\alpha_{k}}\frac{1}{E'}N_{k}\sigma_{pot,k}\phi\left(E^{\prime}\right)\mathrm{d}E^{\prime}=\sum\limits_{k}N_{k}\sigma_{pot,k}\frac{1}{E}}\label{eq:NR-mainConclusion1}\end{equation*}

\end{frame}


\begin{frame}
  \subsection{Heterogeneous Slowing Down (Isolated System)}
  \textbf{Heterogeneous Slowing Down (Isolated System)}


Two region neutron balance:
\begin{align*}\Sigma_{t,f}(E)\phi_{f}(E)V_{f}=P_{f\rightarrow f}(E)V_{f}&\int_{0}^{\infty}\Sigma_{s,f}\left(E^{\prime}\rightarrow E\right)\phi_{f}\left(E^{\prime}\right)dE^{\prime}\\ + P_{m\rightarrow f}(E)V_{m}&\int_{0}^{\infty}\Sigma_{s,m}\left(E^{\prime}\rightarrow E\right)\phi_{m}\left(E^{\prime}\right)dE^{\prime}\end{align*}

  Break apart macroscopic cross sections and substitute in the the probability of energy change via scattering
\begin{equation*}P(E'\rightarrow E)=\frac{1}{(1-\alpha)E}~\mbox{for }\alpha E\leq E'\leq E,\end{equation*}

\begin{align*}\Sigma_{t,f}(E)\phi_{f}(E)V_{f} = P_{f\rightarrow f}(E)V_{f}\sum\limits_{k\in f}&\int_{E}^{E/\alpha_{k}}\frac{N_{k}\sigma_{s,k}\left(E^{\prime}\right)\phi_{f}\left(E^{\prime}\right)}{\left(1-\alpha_{k}\right)E^{\prime}}dE^{\prime}  \\
+ P_{m\rightarrow f}(E)V_{m}\sum\limits_{k\in m}&\int_{E}^{E/\alpha_{k}}\frac{N_{k}\sigma_{s,k}\left(E^{\prime}\right)\phi_{m}\left(E^{\prime}\right)}{\left(1-\alpha_{k}\right)E^{\prime}}dE^{\prime}.\end{align*}

\end{frame}
\begin{frame}
\begin{align*}\Sigma_{t,f}(E)\phi_{f}(E)V_{f} = P_{f\rightarrow f}(E)V_{f}\sum\limits_{k\in f}&\int_{E}^{E/\alpha_{k}}\frac{N_{k}\sigma_{s,k}\left(E^{\prime}\right)\phi_{f}\left(E^{\prime}\right)}{\left(1-\alpha_{k}\right)E^{\prime}}dE^{\prime}  \\
+ P_{m\rightarrow f}(E)V_{m}\sum\limits_{k\in m}&\int_{E}^{E/\alpha_{k}}\frac{N_{k}\sigma_{s,k}\left(E^{\prime}\right)\phi_{m}\left(E^{\prime}\right)}{\left(1-\alpha_{k}\right)E^{\prime}}dE^{\prime}.\end{align*}


Recall from before
  \begin{equation*}{\sum\limits_{k}\frac{1}{1-\alpha_{k}}\int_{E}^{E/\alpha_{k}}\frac{1}{E'}N_{k}\sigma_{pot,k}\phi\left(E^{\prime}\right)\mathrm{d}E^{\prime}=\sum\limits_{k}N_{k}\sigma_{pot,k}\frac{1}{E}}\label{eq:NR-mainConclusion1}\end{equation*}

which helps us simplify the heterogeneous balance equation into

\begin{equation*}\Sigma_{t,f}(E)\phi_{f}(E)V_{f}=\frac{1}{E}\Big(P_{f\rightarrow f}(E)V_{f}\Sigma_{pot,f}+P_{m\rightarrow f}(E)V_{m}\Sigma_{pot,m}\Big)\end{equation*}
\begin{equation*}\phi_{f}(E)=\frac{P_{f\rightarrow f}(E)V_f\Sigma_{pot,f}+P_{m\rightarrow f}(E)V_m\Sigma_{pot,m}}{E\Sigma_{t,f}(E)V_f}\end{equation*}

  \end{frame}
  \begin{frame}
    While this result is derived using a two-region problem, it can be extended to solve for a flux in region $i\in N$, which is dependent on all $j\in N$ regions:

    \begin{equation*}\phi_{i}(E)=\frac{1}{E\Sigma_{t,i(E)}V_i}\sum\limits_j\Big(P_{j\rightarrow i}(E)V_{j}\Sigma_{p,j}\Big)\end{equation*}

  Remember that we used the NR approximation to get here, which could be a source of error in the lower end of the energy spectrum.
  \end{frame}




\section{Tone's Method}
\subsection{Derivation}
  \begin{frame}{Tone's Method}

    \begin{equation*}\phi_{i}(E)=\frac{1}{E\Sigma_{t,i(E)}V_i}\sum\limits_j\Big(P_{j\rightarrow i}(E)V_{j}\Sigma_{p,j}\Big)\end{equation*}
  Crucial approximation for Tone's Method
    \begin{equation*}\frac{P_{j\rightarrow i}(E)}{\Sigma_{t,i}(E)}=\alpha_{i}(E)\frac{P_{j\rightarrow i,g}}{\Sigma_{t,i,g}}\end{equation*}\\[4pt]
      Allow $P_{j\rightarrow i}(E)$ and $\Sigma_{t,i}(E)$ to be constant within a group, but allow a fine energy term $\alpha$\\[4pt]
    Fun twist: $\alpha_i(E)$ is only dependent on the region $i$ that our neutrons are going into

  \end{frame}
    \begin{frame}
      \begin{equation*}\phi_{i}(E)=\frac{\alpha_i(E)}{E\Sigma_{t,i,g}V_i}\sum\limits_j\Big(P_{j\rightarrow i,g}V_{j}\Sigma_{p,j}\Big)\end{equation*}

  We want more information about $\phi_i(E)$. Doing so requires two additional tools:
%We'll use two tools to aid us in the derivation
  \begin{enumerate}
    \item Reciprocity relation
      \begin{equation*}P_{j\rightarrow i}(E)V_{j}\Sigma_{t,j}(E)=P_{i\rightarrow j}(E)V_{i}\Sigma_{t,i}(E)\end{equation*}
        \begin{equation*}P_{i\rightarrow j}(E)=\frac{P_{j\rightarrow i}(E)V_{j}\Sigma_{t,j}(E)}{V_{i}\Sigma_{t,i}(E)}\end{equation*}
    \item Probabilities normalize to 1
      \begin{equation*}\sum\limits_{j}P_{i\rightarrow j}(E)=1\end{equation*}
  \end{enumerate}


Plug reciprocity relation into probabilities requirement
      \begin{equation*}\sum\limits_{j}\left(\frac{P_{j\rightarrow i}(E)V_{j}\Sigma_{t,j}(E)}{V_{i}\Sigma_{t,i}(E)}\right)=1\end{equation*}

    \end{frame}
    \begin{frame}
      \begin{equation*}\sum\limits_{j}\left(\frac{P_{j\rightarrow i}(E)V_{j}\Sigma_{t,j}(E)}{V_{i}\Sigma_{t,i}(E)}\right)=1\end{equation*}

    Plug in the Tone's approximation
    \begin{equation*}\frac{P_{j\rightarrow i}(E)}{\Sigma_{t,i}(E)}=\alpha_{i}(E)\frac{P_{j\rightarrow i,g}}{\Sigma_{t,i,g}}\end{equation*}
      to yield
      \begin{equation*}\frac{\alpha_i(E)}{V_i\Sigma_{t,i,g}}\sum\limits_{j}\Big(P_{j\rightarrow i,g}V_{j}\Sigma_{t,j}(E)\Big)=1\end{equation*}
        \begin{equation*}\alpha_i(E)=\frac{V_i\Sigma_{t,i,g}}{\sum\limits_{j}\Big(P_{j\rightarrow i,g}V_{j}\Sigma_{t,j}(E)\Big)}\end{equation*}

          We can now plug this definition of $\alpha_i(E)$ into our earlier equation for $\phi_i(E)$.
    \end{frame}





    \begin{frame}
      \begin{equation*}\phi_{i}(E)=\frac{\alpha_i(E)}{E\Sigma_{t,i,g}V_i}\sum\limits_j\Big(P_{j\rightarrow i,g}V_{j}\Sigma_{p,j}\Big)\end{equation*}

        \begin{equation*}\phi_{i}(E)=\frac{1}{E\Sigma_{t,i,g}V_i}\frac{V_i\Sigma_{t,i,g}}{\sum\limits_{j}\left(P_{j\rightarrow i,g}V_{j}\Sigma_{t,j}(E)\right)}\sum\limits_j\Big(P_{j\rightarrow i,g}V_{j}\Sigma_{p,j}\Big)\end{equation*}
          \begin{equation*}\phi_{i}(E)=\frac{1}{E}\frac{\sum\limits_j\Big(P_{j\rightarrow i,g}V_{j}\Sigma_{p,j}\Big)}{\sum\limits_{j}\left(P_{j\rightarrow i,g}V_{j}\Sigma_{t,j}(E)\right)}\end{equation*}
            \begin{equation*}\phi_{i}(E)\approx\frac{1}{E}\frac{\sum\limits_j\left(P_{j\rightarrow i,g}V_{j}\left(N_{r,j}\sigma_{pot,r}+\sum\limits_{k\neq r}N_{k,j}\sigma_{pot,k}\right)\right)}{\sum\limits_{j}\left(P_{j\rightarrow i,g}V_{j}\left(N_{r,j}\sigma_{r,t}(E)+\sum\limits_{k\neq r}N_{k,j}\sigma_{pot,k}\right)\right)}\end{equation*}


    \end{frame}









    \begin{frame}
%            \begin{equation*}\phi_{i}(E)=\frac{1}{E}\frac{\sum\limits_j\left(P_{j\rightarrow i,g}V_{j}\left(N_{r,j}\sigma_{pot,r}+\sum\limits_{k\neq r}N_{k,j}\sigma_{pot,k}\right)\right)}{\sum\limits_{j}\left(P_{j\rightarrow i,g}V_{j}\left(N_{r,j}\sigma_{r,t}(E)+\sum\limits_{k\neq r}N_{k,j}\sigma_{pot,k}\right)\right)}\end{equation*}
\begin{equation*}\phi_{i}(E)=\frac{1}{E}\frac{\sum\limits_j\left(P_{j\rightarrow i,g}V_{j}N_{r,j}\sigma_{pot,r}+P_{j\rightarrow i,g}V_{j}\sum\limits_{k\neq r}N_{k,j}\sigma_{pot,k}\right)}{\sum\limits_{j}\left(P_{j\rightarrow i,g}V_{j}N_{r,j}\sigma_{r,t}(E)+P_{j\rightarrow i,g}V_{j}\sum\limits_{k\neq r}N_{k,j}\sigma_{pot,k}\right)}\end{equation*}
\begin{equation*}\phi_{i}(E)=\frac{1}{E}\frac{\sigma_{pot,r}\sum\limits_jP_{j\rightarrow i,g}V_{j}N_{r,j}+\sum\limits_jP_{j\rightarrow i,g}V_{j}\sum\limits_{k\neq r}N_{k,j}\sigma_{pot,k}}{\sigma_{r,t}(E)\sum\limits_{j}P_{j\rightarrow i,g}V_{j}N_{r,j}+\sum\limits_{j}P_{j\rightarrow i,g}V_{j}\sum\limits_{k\neq r}N_{k,j}\sigma_{pot,k}}\end{equation*}
\begin{equation*}\phi_{i}(E)=\frac{1}{E}\frac{\sigma_{pot,r}+\left(\sum\limits_jP_{j\rightarrow i,g}V_{j}\sum\limits_{k\neq r}N_{k,j}\sigma_{pot,k}\Big/\sum\limits_jP_{j\rightarrow i,g}V_{j}N_{r,j}\right)}{\sigma_{r,t}(E)+\Big(\sum\limits_{j}P_{j\rightarrow i,g}V_{j}\sum\limits_{k\neq r}N_{k,j}\sigma_{pot,k}\Big/\sum\limits_{j}P_{j\rightarrow i,g}V_{j}N_{r,j}\Big)}\end{equation*}



    \end{frame}



    \begin{frame}
\begin{equation*}\phi_i(E)=\frac{1}{E}\frac{\sigma_{pot,r}+\sigma_{0}}{\sigma_{t,r}(E)+\sigma_{0}}\end{equation*}

\begin{equation*}\sigma_{0}=\frac{\sum\limits_j\sum\limits_{k\neq r}P_{j\rightarrow i,g}V_{j}N_{k,j}\sigma_{pot,k}}{\sum\limits_jP_{j\rightarrow i,g}V_{j}N_{r,j}}\end{equation*}

\end{frame}

\subsection{Recipe}
    \begin{frame}{Tone's Method}
      \begin{enumerate}
        \item Assume initial background cross sections for resonance nuclides, using conventional equivalence methods
        \item Evaluate the effective cross sections of resonance nuclides using the conventional equivalence theory
        \item Evaluate group-wise collision probability using effective cross sections 
        \item Update the background cross section using
\begin{equation*}\phi_i(E)=\frac{1}{E}\frac{\sigma_{pot,r}+\sigma_{0}}{\sigma_{t,r}(E)+\sigma_{0}}\end{equation*}

\begin{equation*}\sigma_{0}=\frac{\sum\limits_j\sum\limits_{k\neq r}P_{j\rightarrow i,g}V_{j}N_{k,j}\sigma_{pot,k}}{\sum\limits_jP_{j\rightarrow i,g}V_{j}N_{r,j}}\end{equation*}


        \item Repeat until convergence. A few iterations are usually sufficient to obtain the converged result.
      \end{enumerate}
\end{frame}

    \begin{frame}
      \begin{enumerate}
        \item \textbf{Assume initial background cross sections for resonance nuclides, using conventional equivalence methods}
        \item Evaluate the effective cross sections of resonance nuclides using the conventional equivalence theory
        \item Evaluate group-wise collision probability using effective cross sections 
        \item Update the background cross section using
\begin{equation*}\phi_i(E)=\frac{1}{E}\frac{\sigma_{pot,r}+\sigma_{0}}{\sigma_{t,r}(E)+\sigma_{0}}\end{equation*}

\begin{equation*}\sigma_{0}=\frac{\sum\limits_j\sum\limits_{k\neq r}P_{j\rightarrow i,g}V_{j}N_{k,j}\sigma_{pot,k}}{\sum\limits_jP_{j\rightarrow i,g}V_{j}N_{r,j}}\end{equation*}


        \item Repeat until convergence. A few iterations are usually sufficient to obtain the converged result.
      \end{enumerate}
\end{frame}


\begin{frame}
  For a heterogeneous system, our background cross section is comprised of a material-component and a geometry-component.
  \[\sigma_{0,r}=\sigma_{0,f}+\frac{\Sigma_e}{N_r}=\sum_{k\neq r}\frac{N_k\sigma_{s,k}}{N_r}+\frac{\Sigma_e}{N_r}\]
Note that for Tone's method, the initial estimate for background cross section is not too important since it'll iterate out anyway.
\end{frame}

    \begin{frame}
      \begin{enumerate}
        \item Assume initial background cross sections for resonance nuclides, using conventional equivalence methods
        \item \textbf{Evaluate the effective cross sections of resonance nuclides using the conventional equivalence theory}
        \item Evaluate group-wise collision probability using effective cross sections 
        \item Update the background cross section using
\begin{equation*}\phi_i(E)=\frac{1}{E}\frac{\sigma_{pot,r}+\sigma_{0}}{\sigma_{t,r}(E)+\sigma_{0}}\end{equation*}

\begin{equation*}\sigma_{0}=\frac{\sum\limits_j\sum\limits_{k\neq r}P_{j\rightarrow i,g}V_{j}N_{k,j}\sigma_{pot,k}}{\sum\limits_jP_{j\rightarrow i,g}V_{j}N_{r,j}}\end{equation*}


        \item Repeat until convergence. A few iterations are usually sufficient to obtain the converged result.
      \end{enumerate}
\end{frame}



\begin{frame}
  Use GROUPR to create a table of cross sections vs. dilution for my resonant nuclides.
\begin{figure}

\includegraphics[width=0.75\textwidth]{gendf}
\end{figure}


\end{frame}

    \begin{frame}
      \begin{enumerate}
        \item Assume initial background cross sections for resonance nuclides, using conventional equivalence methods
        \item Evaluate the effective cross sections of resonance nuclides using the conventional equivalence theory
        \item \textbf{Evaluate group-wise collision probability using effective cross sections }
        \item Update the background cross section using
\begin{equation*}\phi_i(E)=\frac{1}{E}\frac{\sigma_{pot,r}+\sigma_{0}}{\sigma_{t,r}(E)+\sigma_{0}}\end{equation*}

\begin{equation*}\sigma_{0}=\frac{\sum\limits_j\sum\limits_{k\neq r}P_{j\rightarrow i,g}V_{j}N_{k,j}\sigma_{pot,k}}{\sum\limits_jP_{j\rightarrow i,g}V_{j}N_{r,j}}\end{equation*}


        \item Repeat until convergence. A few iterations are usually sufficient to obtain the converged result.
      \end{enumerate}
\end{frame}





\begin{frame}
  Created a quick Monte Carlo script that solves for 1 group collision probabilities 
\begin{figure}
\includegraphics[width=0.7\textwidth]{collisionProb1}
\end{figure}
  Geometry: 3x3 grid, with and without a center hole. Reflective bounds



\end{frame}



    \begin{frame}
      \begin{enumerate}
        \item Assume initial background cross sections for resonance nuclides, using conventional equivalence methods
        \item Evaluate the effective cross sections of resonance nuclides using the conventional equivalence theory
        \item Evaluate group-wise collision probability using effective cross sections 
        \item \textbf{Update the background cross section using}
\begin{equation*}\phi_i(E)=\frac{1}{E}\frac{\sigma_{pot,r}+\sigma_{0}}{\sigma_{t,r}(E)+\sigma_{0}}\end{equation*}

\begin{equation*}\sigma_{0}=\frac{\sum\limits_j\sum\limits_{k\neq r}P_{j\rightarrow i,g}V_{j}N_{k,j}\sigma_{pot,k}}{\sum\limits_jP_{j\rightarrow i,g}V_{j}N_{r,j}}\end{equation*}


        \item Repeat until convergence. A few iterations are usually sufficient to obtain the converged result.
      \end{enumerate}
\end{frame}



\section{Results}
\begin{frame}
\begin{figure}
\includegraphics[width=0.7\textwidth]{sig0Estimations}
  \caption{Tone's method evolving background cross section $\sigma_0$ across iterations}
\end{figure}

\end{frame}



\begin{frame}
\begin{figure}
\includegraphics[width=0.8\textwidth]{final_3x3}
\end{figure}
\end{frame}


\begin{frame}
\begin{figure}
\includegraphics[width=0.8\textwidth]{final_3x3_with_hole}
\end{figure}
\end{frame}

\begin{frame}
\begin{figure}
\includegraphics[width=0.8\textwidth]{error_3x3}
\end{figure}
\end{frame}


\begin{frame}
\begin{figure}
\includegraphics[width=0.8\textwidth]{error_3x3_with_hole}
\end{figure}
\end{frame}

\begin{frame}
\begin{figure}
\includegraphics[width=0.8\textwidth]{error_3x3_with_without_hole}
\end{figure}
\end{frame}




\begin{frame}
\end{frame}






\begin{frame}
\end{frame}






\begin{frame}
\end{frame}





\begin{frame}
\end{frame}









\end{document}
